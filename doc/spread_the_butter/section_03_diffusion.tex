% !TEX root = ./main.tex
\section{Implementing the evolutionary forces}

As hinted in Fig.~\ref{fig01:moran}(A), the Moran process gives us a simple
recipe on how to encode the different evolutionary forces on the transition
rates $W^\pm(x)$. Recall that since the transition rates in allele frequency are
the same as the transition rates in number of organisms, we can more easily
conceptualize the forces in the latter space. This means that the general form
in which the number of organisms with allele $A$ can chance looks like
\begin{equation}
    W^+(x) = w^+(n) = 
    \text{Rate of $A$ dying} \times
    \text{Probability of replacement by $a$},
\end{equation}
and 
\begin{equation}
    W^-(x) = w^-(n) = 
    \text{Rate of $a$ dying} \times
    \text{Probability of replacement by $A$},
\end{equation}
Having these general forms let us define the rates for different evolutionary
forces.

\subsection{Genetic drift}

Given the intrinsic stochasticity of the Moran process, the easiest evolutionary
force to implement is that of genetic drift. For simplicity, we assume that both
allele types have the same death rate per organism $\gamma$. For the case where
only genetic drift is changing the population composition, we assume that the
reproduction probability per organism is the same (equivalent to saying both
genotypes have the same fitness), so the probability of replacement is given by
the allele frequency. Mathematically this means that we can express the rate
with which the number of organisms with allele $A$ increases as
\begin{equation}
    W^+(x) = w^+(n) = \gamma (N-n) \frac{n}{N},
\end{equation}
where the term $\gamma (N-n)$ defines the rate at which an organism with an
allele $a$ dies, and the term $n/N$ defines the probability of an organism with
allele $A$ reproducing. Equivalently the rate with which the number of organisms
with allele $a$ decreases takes the form
\begin{equation}
    W^+(x) = w^+(n) = \gamma n \frac{(N-n)}{N}.
\end{equation}
Given that both rates are equal, we can see that the first term in
Eq.~\ref{eq:master_x} involving $W^+(x) - W^-(x)$ is zero. Substituting the sum
of the rates into the second term of Eq.~\ref{eq:master_x} results in
\begin{equation}
\begin{aligned}
    \frac{\partial P(x, t)}{\partial t} &=\frac{1}{2 N^{2}} 
    \frac{\partial^{2}}{\partial x^{2}}[2 \gamma(N-n) n P(x, t)], \\
    &=\frac{\partial^{2}}{\partial x^{2}}
    \left[\frac{1}{2 N^{2}} 2 \gamma\left(\frac{N-n) n}{N} \right) P(x, t)\right],\\
    &=\frac{\gamma}{N} \frac{\partial^{2}}{\partial x^{2}}[x(1-x) P(x, t)].
\end{aligned}
\end{equation}
We redefine the time scale to be in units of $\gamma^{-1}$, allowing us to write
\begin{equation}
    \frac{\partial P(x, t)}{\partial t} =
    \frac{1}{N} \frac{\partial^{2}}{\partial x^{2}}[x(1-x) P(x, t)],
\end{equation}
the classic Kimura diffusion equation for genetic drift only.