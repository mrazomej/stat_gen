% !TEX root = ./main.tex
\section{Implementing the evolutionary forces}

As hinted in Fig.~\ref{fig01:moran}(A), the Moran process gives us a simple
recipe on how to encode the different evolutionary forces on the transition
rates $W^\pm(x)$. Recall that since the transition rates in allele frequency are
the same as the transition rates in number of organisms, we can more easily
conceptualize the forces in the latter space. This means that the general form
in which the number of organisms with allele $A$ can chance looks like
\begin{equation}
    W^+(x) = w^+(n) = 
    \text{Rate of $A$ dying} \times
    \text{Probability of replacement by $a$},
\end{equation}
and 
\begin{equation}
    W^-(x) = w^-(n) = 
    \text{Rate of $a$ dying} \times
    \text{Probability of replacement by $A$},
\end{equation}
Having these general forms let us define the rates for different evolutionary
forces.

\subsection{Genetic drift}
\mrm{Further discussion of what genetic drift is?}

Given the intrinsic stochasticity of the Moran process, the easiest evolutionary
force to implement is that of genetic drift. For simplicity, we assume that both
allele types have the same death rate per organism $\gamma$. For the case where
only genetic drift is changing the population composition, we assume that the
reproduction probability per organism is the same (equivalent to saying both
genotypes have the same fitness), so the probability of replacement is given by
the allele frequency. Mathematically this means that we can express the rate
with which the number of organisms with allele $A$ increases as
\begin{equation}
    W^+(x) = w^+(n) = \gamma (N-n) \frac{n}{N},
\end{equation}
where the term $\gamma (N-n)$ defines the rate at which an organism with an
allele $a$ dies, and the term $n/N$ defines the probability of an organism with
allele $A$ reproducing. Equivalently the rate with which the number of organisms
with allele $a$ decreases takes the form
\begin{equation}
    W^+(x) = w^+(n) = \gamma n \frac{(N-n)}{N}.
\end{equation}
Given that both rates are equal, we can see that the first term in
Eq.~\ref{eq:master_x} involving $W^+(x) - W^-(x)$ is zero. Substituting the sum
of the rates into the second term of Eq.~\ref{eq:master_x} results in
\begin{equation}
\begin{aligned}
    \frac{\partial P(x, t)}{\partial t} &=\frac{1}{2 N^{2}} 
    \frac{\partial^{2}}{\partial x^{2}}[2 \gamma(N-n) n P(x, t)], \\
    &=\frac{\partial^{2}}{\partial x^{2}}
    \left[\frac{1}{2 N^{2}} 2 \gamma\left(\frac{N-n) n}{N} \right) P(x, t)\right],\\
    &=\frac{\gamma}{N} \frac{\partial^{2}}{\partial x^{2}}[x(1-x) P(x, t)].
\end{aligned}
\end{equation}
We redefine the time scale to be in units of $\gamma^{-1}$, allowing us to write
\begin{equation}
    \frac{\partial P(x, t)}{\partial t} =
    \frac{1}{N} \frac{\partial^{2}}{\partial x^{2}}[x(1-x) P(x, t)],
\end{equation}
the classic Kimura diffusion equation for genetic drift only.

\subsection{Genetic drift plus selection}

Natural selection is intrinsically associated with the concept of fitness. The
phrase ``survival of the fittest'' first used by Darwin guided and still guides
the way that biologists think about the evolution of many organisms. But
despite the fact that fitness is part of the daily jargon of many biologists,
it is a subtle and highly debated concept. After all what defines the ability
of an organism to survive the challenges that surround them are completely
context dependent. Roughly speaking we can think of fitness as the ability of
an organism, or a population of organisms to survive and reproduce in the given
ecological niche they occupy. The term ecology has to be included because
fitness is a result of the interplay between organisms with their environment
including all biotic and abiotic interactions. 

It is common both in theory and in experiments to use the relative growth rates
of organisms, i.e. the speed at which they can reproduce and generate
offspring, as a proxy for fitness. This is a convenient approximation both
for experiments and for theory, but one should not lose track of the relevant
context dependence on the fitness. Just because redwoods have an average life
span of 500-700 years and a very low growth rate that doesn't mean they are not
fit. Having said that we will first begin with the simplest form of fitness,
i.e. frequency independent selection. The term frequency independence simply
refers to the assumption that the fitness of a particular allele does not
depend on the relative abundance of such allele. This assumption could break
down for cases such as some pathogenic bacteria that coordinate their attack
via cell-to-cell communication known as quorum sensing. 

To implement the effect of different reproductive success for different alleles
we introduce parameters $f_A$ and $f_a$ as the fitness values for allele $A$ and
$a$, respectively. With these parameters in hand we must redefine the
probability of an organism reproducing to replace the one that dies in the Moran
process. The replacement probability for allele $A$ is now given by
\begin{equation}
    \text{Prob. of $A$ replacing} = \frac{f_A n}{f_A n + f_a (N - n)}.
\end{equation}
Likewise for allele $a$ we have
\begin{equation}
    \text{Prob. of $a$ replacing} = \frac{f_a (N - n)}{f_A n + f_a (N - n)}.
\end{equation}
Let us now assume that $f_A \approx (1 + s) f_a$ for a small $s$. This parameter
$s$ is the so-called selection coefficient, which in nature can be of the order
of $10^{-3}$ or less. With these assumptions we can simplify these substitution
probabilities to be
\begin{equation}
    \text{Prob. of $A$ replacing} \approx \frac{n}{N}(1 + s),
\end{equation}
and
\begin{equation}
    \text{Prob. of $a$ replacing} \approx \frac{(N - n)}{N}.
\end{equation}
This small change in the replacement probability results in transition rates for
Eq.~\ref{eq:master_x} of the form
\begin{equation}
    W^+(x) = \gamma (N - n) \frac{n}{N} (1 + s),
\end{equation}
and
\begin{equation}
    W^-(x) = \gamma n \frac{(N - n)}{N}.
\end{equation}
The difference of these two rates takes the form
\begin{equation}
    W^+(x) - W^-(x) = \gamma s \frac{n(N-n)}{N}.
\end{equation}
The sum results in
\begin{equation}
    W^+(x) + W^-(x) = \gamma (2 + s) \frac{n(N-n)}{N}.
\end{equation}
Substituting this into Eq.~\ref{eq:master_x} results in
\begin{equation}
    \frac{\partial}{\partial t} P(x, t) =
    -\frac{1}{N} 
    \frac{\partial}{\partial x}\left[\gamma s \frac{(N-n) n}{N} P(x, t)\right] 
    +\frac{1}{2 N^{2}} 
    \frac{\partial^{2}}{\partial x^{2}}
    \left[\gamma (2+s)\left(\frac{N-n) n}{N}\right) P(x, t)\right].
\end{equation}
Simplifying terms and substituting the definition of the allele frequency gives
\begin{equation}
    \frac{\partial}{\partial t} P(x, t) =
    -\gamma_{S} \frac{\partial}{\partial x}[x(1-x) P(x, t)] 
    +\frac{\gamma\left(1+\frac{s}{2}\right)}{N}
    \frac{\partial^{2}}{\partial x^{2}}[x(1-x) P(x, t)].
\end{equation}
To obtain the final form we again write the time scale in units of
$\gamma^{-1}$. Furthermore we use the simplification that $s \ll 1$, obtaining
the classic Kimura diffusion equation for selection and drift
\begin{equation}
    \frac{\partial}{\partial t} P(x, t) =
    -\frac{\partial}{\partial x}[s x(1-x) P(x, t)] 
    +\frac{1}{N} \frac{\partial^{2}}{\partial x^{2}}[x(1-x) P(x, t)].
\end{equation}

\subsection{Genetic drift plus selection plus mutation}

One of the ingredients for evolution to take place is the constant appearance of
genetic variability. After all, the raw material for evolution to act on is the
appearance of new mutations in the population. The implementation of this third
force changes the possibilities on how to change the composition of the
population. As depicted in Fig.~\ref{fig01:moran}(A), if mutation is taken into
account, there are two possible substitutions which would modify the allele
frequency: 1. The usual path in which an organism with the opposite allele to
the one that died reproduces and does not mutate when doing so, and 2. the
possibility of an organism of the same allele as the one that died reproducies,
but when doing so, it mutates to the opposite allele. For simplicity we will
assume that the mutation probability from $A$ to $a$, $\mu_{A\rightarrow a}$, is
the same as from $a$ to $A$, $\mu_{a\rightarrow A}$. The state transition rate
to increase the allele frequency now take the form
\begin{equation}
    W^{+}(x) = \gamma(N-n) \frac{n}{N}(1+s)(1-\mu)+\gamma(N-n) \frac{(N-n)}{N} \mu,
\end{equation}
where the factor of $(1 - \mu)$ represents the probability of not mutating.
Equivalently for the rate that decreases the allele frequency we have
\begin{equation}
    W^{-}(x)=\gamma n \frac{(N-n)}{N}(1-\mu)+\gamma n \frac{n}{N}(1+s) \mu.
\end{equation}
After some algebra, we find that the difference between these rates is of the
form
\begin{equation}
    W^+(x) - W^-(x) = 
    \frac{\gamma}{N}\left[n s\left(N-n-N\mu\right)+N{\mu}(N-2 n)\right].
\end{equation}
For the sum of the rates we find
\begin{equation}
    W^+(x) + W^-(x) =
    \frac{\gamma}{N}\left[N n(2-4 \mu+s-\mu s) + 
    n^{2}(-2+4 \mu-s+2 \mu s)\right].
\end{equation}
Substituting these rates in Eq.~\ref{eq:master_x} gives
\begin{equation}
\begin{split}
    \frac{\partial}{\partial t} P(x, t)=
    &-\frac{1}{N} \frac{\partial}{\partial x}
    \left[\frac{\gamma}{N}\left(n s\left(N-n-N\mu\right)+
    N\mu(N-2 n)\right) P(x, t)\right] \\
    &+\frac{1}{2 N^{2}} \frac{\partial^2}{\partial x^{2}}
    \left[\frac{\gamma}{N}\left(N_{n}(z-4 \mu+s-\mu s)
    -n^{2}\left(z-4 \mu+s-2\mu s\right)\right) P(x, t)\right].
\end{split}
\end{equation}
Substituting the definition of the allele frequency results in
\begin{equation}
\begin{split}
    \frac{\partial}{\partial t} P(x, t)=
    &-\gamma \frac{\partial}{\partial x}[x s(1-x-\mu)+\mu(1-2 x) P(x, t)] \\
    &+\frac{\gamma}{2 N} \frac{\partial^{2}}{\partial x}
    \left[x(2-4 \mu+s-\mu s)-x^{2}(2-4 \mu+s-2 \mu s) P(x, t)\right].
\end{split}
\end{equation}
To get to the final equation we simply make use of the approximation that both
$s, \mu \ll 1$. Implementing this, and writing the time scale in units of
$\gamma^{-1}$ results in the classic diffusion theory equation with all three
forces implemented
\begin{equation}
    \frac{\partial}{\partial t} P(x, t) =
    -\frac{\partial}{\partial x}[s x(1-x) + \mu (1 - 2x) P(x, t)] 
    +\frac{1}{N} \frac{\partial^{2}}{\partial x^{2}}[x(1-x) P(x, t)].
\end{equation}