% !TEX root = ./main.tex
\section{Stochastic reproduction and the Moran process}

Our approach to derive the equation that governs the time evolution of the
allele frequency starts from a discrete state space. As mentioned before, we
will be working in the context of the one-locus two-allele model. To capture the
stochastic nature of the reproduction of these organisms we will adopt the Moran
process, shown schematically in Fig.~\ref{fig01:moran}(A). The Moran process is
a stochastic process with a fix population size $N$. Organisms in this discrete
population die with a characteristic rate $\gamma$. The type of organism that
dies during a small time window $\Delta t$ depends on the relative abundance of
each allele type. The organism that dies is immediately replaced when any of the
organisms in the population reproduces. The probability of which organism gets
to reproduce can depend on the relative fitness values of each allele.
Furthermore, when the organism reproduces, there is an intrinsic probability
that the progeny mutates to the opposite allele.

\begin{figure}[h!]
	\centering \includegraphics {../../fig/spread_the_butter/fig01_moran.pdf}
  \caption{\textbf{Schematic of the Moran process}. (A) In the Moran process the
  population size $N$ remains constant at all times. Organisms stochastically
  die at a rate $\gamma$ to be immediately replaced by another organism that is
  able to reproduce. For the composition of the population to change two things
  can happen: 1. An organism of the same type than the one that died reproduces
  --with a probability depending on the fitness--,and then this organism mutates
  to the opposite allele type, or 2. An organism of the opposite allele
  reproduces--also with a probability depending on the fitness--, and this
  organism does not mutate when doing so. (B) Schematic representation of the
  construction of the master equation for the number of organisms carrying
  allele $A$. On a small discrete time step $\Delta t$ the proabbility of
  transitioning to from a number $n$ to a number $n + 1$ or $n-1$ is given by
  $w^+(n)$ and $w^-(n)$, respectively. (C) In the limit of a large population we
  can approximate the probability mass function for the number of organisms
  $p(n, t)$ with a probability density function for the allele frequency $P(x,
  t)$.}
  \label{fig01:moran}
\end{figure}

\subsection{General master equation for the Moran process}

Given the discrete nature of the Moran process, our first task consists of
writing the probability mass function (PMF) for the number of organisms carrying
allele $A$. Fig~\ref{fig01:moran}(B) shows a schematic of how to build
such distribution. Focusing on a particular number of organisms $n$, during a
sufficiently small time window $\Delta t$ we must take into account all of the
inflow and outflow of probability from this particular bin. Let us define
$W^+(n)$ and $W^-(n)$ to be the transition rates (in units of time$^{-1}$) of
jumping from bin $n$ to bin $n+1$ and from bin $n$ to bin $n-1$, respectively.
Writing down the checks and balances during this small time window allows us to
predict the probability of having a particular number of organisms. This is
mathematically written as
\begin{equation}
    p(n, t + \Delta t) = 
    p(n, t)
    + \overbrace{w^+(n - 1)\Delta t p(n - 1, t)}^
    {n - 1 \rightarrow n}
    + \overbrace{w^-(n + 1)\Delta t p(n + 1, t)}^
    {n + 1 \rightarrow n}
    - \overbrace{w^-(n)\Delta t p(n, t)}^
    {n \rightarrow n - 1}
    - \overbrace{w^+(n)\Delta t p(n, t)}^
    {n \rightarrow n + 1}.
    \label{eq:master_n_discrete}
\end{equation}
It is in the definition of these rates $w^\pm(n)$ that we include the effects
that difference evolutionary forces have on the probability of changing the
composition of the population as schematized in Fig.~\ref{fig01:moran}(A).

We can further simplify Eq.~\ref{eq:master_n_discrete} by sending the first term
on the right hand side to the left, dividing both sides by $\Delta t$, and
taking the limit $\Delta t \rightarrow 0$. This results in the so-called master
equation for the number of organisms carrying allele $A$
\begin{equation}
    \frac{dp(n, t)}{dt} = 
    w^+(n - 1) p(n - 1, t)
    + w^-(n + 1) p(n + 1, t)
    - w^+(n) p(n, t)
    - w^-(n) p(n, t).
    \label{eq:master_n}
\end{equation}

\subsection{Fokker-Planck equation for allele frequency}

Part of the beauty and power of diffusion theory is that we can work with smooth
continuous equations that track the evolution of the population composition.
Therefore, rather than tracking the discrete number of organisms, we will work
instead with allele frequency. In the limit of a very large populations--one of
the conditions for diffusion theory to apply--we can approximate the allele
frequency $x$ to be a continuous variable between 0 and 1. As schematized in
Fig.~\ref{fig01:moran}(C), we can bin the continuous probability density
function for the allele frequency $P(x, t)$ into small bins of $\Delta x$. The
natural width of such bins is given by the minimum change that can happen in the
discrete language, i.e., $\Delta x \approx 1 / N$. With this definition in hand
we can then approximate the PMF of $n$ as
\begin{equation}
  p(n, t) \approx
  \overbrace{P(x, t)}^{\text{base}} \;
  \overbrace{\Delta x}^{\text{width}}.
\end{equation}
The transition rates $w^\pm(n)$, having units of time$^{-1}$, are not affected
by this approximation. In other words, the transition rate to jump from $n$ to
$n \pm 1$ is the exact same transition rate to jump from $x$ to $x \pm \Delta
x$. We then define the transition rates in allele frequency as
\begin{equation}
  W^\pm(x) \equiv w^\pm(n).
\end{equation}
Substituting these definitions we can rewrite Eq.~\ref{eq:master_n} as
\begin{equation}
\begin{split}
    \frac{dP(x, t) \Delta x}{dt} = 
    &W^+(x - \Delta x) P(x - \Delta x, t) \Delta x
    + W^-(x + \Delta x) P(x + \Delta x, t) \Delta x\\
    &- W^+(x) P(x, t) \Delta x
    - W^-(x) p(x, t) \Delta x.
\end{split}
\end{equation}
Given that $\Delta x$ does not depend on the time $t$ we can take it out of the
derivative and simplify it from both sides, obtaining the master equation for
the time evolution of the probability density function of the allele frequency
\begin{equation}
    \frac{dP(x, t)}{dt} = 
    W^+(x - \Delta x) P(x - \Delta x, t)
    + W^-(x + \Delta x) P(x + \Delta x, t)
    - W^+(x) P(x, t) 
    - W^-(x) p(x, t).
    \label{eq:master_x} 
\end{equation}
Given the continuous nature of Eq.~\ref{eq:master_x} we can Taylor expand the
terms involving $x \pm \Delta x$. The expansion of these terms up to second
order takes the form
\begin{equation}
  W^{\mp}(x \pm \Delta x) P(x \pm \Delta x) \approx
  W^\mp(x) P(x, t) \pm
  \frac{d}{dx} \left[W^\mp(x) P(x, t) \right] \Delta x
  + \frac{1}{2} \frac{d^2}{dx^2} \left[ W^\mp(x) P(x, t) \right] (\Delta x)^2
  + \mathcal{O}((\Delta x)^3)
\end{equation}
Substituting these expansions into Eq.\ref{eq:master_x} results in a partial
differential equation of the form
\begin{equation}
\begin{aligned}
  \frac{\partial}{\partial t} P(x, t) &= 
  W^{+}(x) P(x, t)
  -\frac{\partial}{\partial x}
  \left[W^{+}(x) P(x, t)\right] \Delta x
  +\frac{1}{2} \frac{\partial^{2}}{\partial x^{2}}
  \left[W^{+}(x) P(x, t)\right](\Delta x)^{2} \\
  &+W^{-}(x) P(x, t)+
  \frac{\partial}{\partial x}
  \left[W^{-}(x) P(x, t)\right] \Delta x
  + \frac{1}{2} \frac{\partial^{2}}{\partial x^{2}}
  \left[W^{-(x)} P(x, t)\right](\Delta x)^{2} \\
  &-W^{+}(x, t) P(x, t) -W^{-}(x) P(x, t).
\end{aligned}
\end{equation}
Simplifying terms and using the linearity of the derivatives we can rewrite this
as
\begin{equation}
\frac{\partial}{\partial t} P(x, t)=
-\frac{\partial}{\partial x}
\left[\left(W^{+}(x)-W^{-}(x)\right) P(x, t)\right] \Delta x 
+\frac{1}{2} \frac{\partial^{2}}{\partial x^{2}}
\left[\left(W^{+}(x)+W^{-}(x)\right) P(x, t)\right](\Delta x)^{2}
\end{equation}
Finally, recall that we define $\Delta x \equiv 1 / N$. Substituting this gives
us the partial differential equation in which we will implement the different
evolutionary forces via the transition rates
\begin{equation}
\frac{\partial}{\partial t} P(x, t)=
-\frac{1}{N}\frac{\partial}{\partial x}
\left[\left(W^{+}(x)-W^{-}(x)\right) P(x, t)\right] 
+\frac{1}{2N^2} \frac{\partial^{2}}{\partial x^{2}}
\left[\left(W^{t}(x)+W^{-}(x)\right) P(x, t)\right]
\label{eq:pde_x_general}
\end{equation}

In the following section we will implement one-by-one the three evolutionary
forces via the definition of the transition rates.