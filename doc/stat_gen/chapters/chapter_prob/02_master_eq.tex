\section{The (continuous) master equation for allele frequencies}

One of the most powerful tools to study stochastic process is the famous master
equation. Originally devised to study the stochastic time evolution of chemical
reactions - thereby christened the chemical master equation - this equation has
found applications in many areas of chemistry, physics and biology. The equation
is as statement about the time evolution of the transition probabilities of a
Markov process. That is just a fancy way of saying that the master equation
describes how the transition probabilities between states change over time.

Let's derive this powerful equation starting from \eref{eq_chapman_kolmogorov}.
For our case of study where we want to understand how the frequency of a
particular allele $f$ evolves over time, the Chapman-Kolmogorov equation takes
the form
\begin{equation}
  P(f_3, t_3 \mid f_1, t_1) = \int_0^1 df_2\; P(f_3, t_3 \mid f_2, t_2)
                                          P(f_2, t_2 \mid f_1, t_1),
\end{equation}
where the integration limits $[0, 1]$ are the domain of values that an allele
frequency can take. Now let us assume that we measure the frequency $f_1$ at
time $t$, and then after a very short time $\Delta t$ we measure the second
allele frequency $f_2$. Under this short time limit we can approximate the
transition probability as
\begin{equation}
  P(f_2, t + \Delta t \mid f_1, t) = \delta(f_2 - f_1)
  \underbrace{\left[ 1 - a^{(0)}(f_1, t) \Delta t \right]}_{\text{probability
  of no transition}} +
  \underbrace{\phi_t(f_2 \mid f_1)\Delta t}_\text{probability of transition} +
  \mathcal{O}(\Delta t^2).
  \label{eq_transition_short_time}
\end{equation}
Let's break down this equation. We have split the possible things that can
happen on a time window $\Delta t$ into two possible cases. The first one
represented by the first term is the possibility that on this small time window
no transition actually takes place. The $\delta$-function that is equal to 1 if
and only if $f_2 = f_1$ is there to make sure that this term is added only when
there was no transition during that time and $f_2$ remains the same as $f_1$.
Inside the square brackets we wrote $1 - a^{(0)}(f_1, t) \Delta t$, the reason
for writing the term $a^{(0)}$ will become clear later on when we derive the
Fokker-Planck equation. The fact that we have a term of the form 1 - something
hints at the fact that this ``something'' must be the probability of
transitioning somewhere else rather than staying still. For the second term
we wrote $\phi_t(f_2 \mid f_1)\Delta t$ as the probability of transitioning
outside of $f_1$ during this time window. our function $\phi_t(f_2 \mid f_1)$
represents the transition rate between $f_1$ and $f_2$ at time $t$. When we
multiply this rate in time$^{-1}$ units times a small time window, we obtain the
probability of transitioning from $f_1$ to $f_2$.

In order to understand better the term $a^{(0)}(f_1, t)$ in
\eref{eq_transition_short_time} recall that a probability distribution must be
normalized. That means that if we integrate both sides of
\eref{eq_transition_short_time} it must be true that
\begin{equation}
  1 = \int_0^1 df_2 \; P(f_2, t + \Delta t \mid f_1, t)
\end{equation}
