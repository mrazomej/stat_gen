\subsection{From Langevin dynamics to Fokker-Planck equation}

Now that we built a description for the probability distribution over the
ensemble of trajectories that a Markovian stochastic process can follow we are
left with the task of how to define the Fokker-Plank equation coefficients from
the stochastic dynamics defined by the Langevin equation. Specifically in its
more general form we derived \eref{eq_fokker_planck} to depend on three things:
the probability distribution of allele frequencies $P(f, t)$, and the two jump
moments $a^{(1)}(f, t)$ and $a^{(2)}(f, t)$, where these moments are given by
\eref{eq_jump_mom}. A lot of the population genetics literature use a
non-mathematical argument to directly assign the values of these coefficients
given the either deterministic or stochastic nature of the terms. For example,
given the dynamics that we defined in \secref{seq_determ_mut} for the
deterministic form of mutation and selection (\eref{eq_sel_mut}) most texts
would directly assign these dynamics to be the first jump moment
$a^{(1)}(f,t)$. There is nothing intrinsically wrong with this approach, since
this is actually the correct answer, but it is not necessarily obvious that
this should be the case especially since \eref{eq_fokker_planck} requires us to
take a derivative with respect to $x$ of this coefficient multiplied by the
allele frequency distribution.

We will take a more formal approach and find how the terms in the Langevin
dynamics that we defined in \secref{sec_langevin_intro} connect to the
coefficients of the Fokker-Planck equation. To do so first we introduce a
quantity defined as
\begin{equation}
    \mathcal{A}(x; \tau, t) \equiv \int_{-\infty}^\infty dx\;
    (x - x')^m P(x, t + \tau \mid x', t), \; \text{for } m \geq 1.
    \label{eq_compute_jump_mom}
\end{equation}
This is nothing else than the average of $[X(t + \tau) - X(t)]^m$ with a sharp
initial condition $X(t) = x'$. So this is the average difference between two
time points $t$ and $t + \tau$ of the dynamics defined by our Langevin
dynamics. In other words $\mathcal{A}(y; \tau, t)$ can be expressed as
\begin{equation}
    \mathcal{A}(x; \tau, t) \equiv 
    \ee{\left[X(t + \tau) - X(t)\right]^m \mid X(t) = x'}.
\end{equation}
For small $\tau$ we can rewrite the conditional distribution $P(x, t + \tau
\mid x', t)$ using \eref{eq_transition_short_time}. Using this we can rewrite
\eref{eq_compute_jump_mom} as
\begin{equation}
    \mathcal{A}(x; \tau, t) = \int_{-\infty}^\infty dx\; (x - x')^m
    \underbrace{
    \left[ \delta(x - x') \left( 1 - a^{(0)}(x', t) \tau \right) +
    \phi_t(x \mid x')\tau \right]
    }_{P(x, t + \tau \mid x', t)}.
\end{equation}
Distributing the integrals results in
\begin{align}
    \mathcal{A}(x; \tau, t) &= 
    \int_{-\infty}^\infty dx\; (x - x')^m \delta(x - x') \\
    &- \int_{-\infty}^\infty dx\; (x - x')^m \delta(x - x') a^{(0)}(x', t) 
    \tau \\ 
    &+ \int_{-\infty}^\infty dx\; (x - x')^m \phi_t(x \mid x') \tau.
\end{align}
For the first two terms we have that the $\delta$-function is one only when $x
= x'$ while the term $(x - x')^m$ is zero for this particular case. Therefore
both of these terms cancel and we are only left with
\begin{equation}
    \mathcal{A}(x; \tau, t) = \int_{-\infty}^\infty dx\; (x - x')^m 
    \phi_t(x \mid x') \tau.
\end{equation}
This is non other than the definition of the jump moments times $\tau$ given by
\eref{eq_jump_mom} since we defined $r \equiv x - x'$ and $\phi_t(x'; r) \equiv
\phi_t(x \mid x')$. Now we can see the connection; we started defining
$\mathcal{A}(x; \tau, t)$ to be the average displacement defined by our
Langevin dynamics over a small time interval $\tau$, and we found that this is
connected to the jump moments that define the coefficients of the Fokker-Planck
equation.